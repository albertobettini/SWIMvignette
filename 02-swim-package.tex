\documentclass[]{article}
\usepackage{lmodern}
\usepackage{amssymb,amsmath}
\usepackage{ifxetex,ifluatex}
\usepackage{fixltx2e} % provides \textsubscript
\ifnum 0\ifxetex 1\fi\ifluatex 1\fi=0 % if pdftex
  \usepackage[T1]{fontenc}
  \usepackage[utf8]{inputenc}
\else % if luatex or xelatex
  \ifxetex
    \usepackage{mathspec}
  \else
    \usepackage{fontspec}
  \fi
  \defaultfontfeatures{Ligatures=TeX,Scale=MatchLowercase}
\fi
% use upquote if available, for straight quotes in verbatim environments
\IfFileExists{upquote.sty}{\usepackage{upquote}}{}
% use microtype if available
\IfFileExists{microtype.sty}{%
\usepackage{microtype}
\UseMicrotypeSet[protrusion]{basicmath} % disable protrusion for tt fonts
}{}
\usepackage[margin=1in]{geometry}
\usepackage{hyperref}
\hypersetup{unicode=true,
            pdfborder={0 0 0},
            breaklinks=true}
\urlstyle{same}  % don't use monospace font for urls
\usepackage{longtable,booktabs}
\usepackage{graphicx,grffile}
\makeatletter
\def\maxwidth{\ifdim\Gin@nat@width>\linewidth\linewidth\else\Gin@nat@width\fi}
\def\maxheight{\ifdim\Gin@nat@height>\textheight\textheight\else\Gin@nat@height\fi}
\makeatother
% Scale images if necessary, so that they will not overflow the page
% margins by default, and it is still possible to overwrite the defaults
% using explicit options in \includegraphics[width, height, ...]{}
\setkeys{Gin}{width=\maxwidth,height=\maxheight,keepaspectratio}
\IfFileExists{parskip.sty}{%
\usepackage{parskip}
}{% else
\setlength{\parindent}{0pt}
\setlength{\parskip}{6pt plus 2pt minus 1pt}
}
\setlength{\emergencystretch}{3em}  % prevent overfull lines
\providecommand{\tightlist}{%
  \setlength{\itemsep}{0pt}\setlength{\parskip}{0pt}}
\setcounter{secnumdepth}{5}
% Redefines (sub)paragraphs to behave more like sections
\ifx\paragraph\undefined\else
\let\oldparagraph\paragraph
\renewcommand{\paragraph}[1]{\oldparagraph{#1}\mbox{}}
\fi
\ifx\subparagraph\undefined\else
\let\oldsubparagraph\subparagraph
\renewcommand{\subparagraph}[1]{\oldsubparagraph{#1}\mbox{}}
\fi

%%% Use protect on footnotes to avoid problems with footnotes in titles
\let\rmarkdownfootnote\footnote%
\def\footnote{\protect\rmarkdownfootnote}

%%% Change title format to be more compact
\usepackage{titling}

% Create subtitle command for use in maketitle
\providecommand{\subtitle}[1]{
  \posttitle{
    \begin{center}\large#1\end{center}
    }
}

\setlength{\droptitle}{-2em}

  \title{}
    \pretitle{\vspace{\droptitle}}
  \posttitle{}
    \author{}
    \preauthor{}\postauthor{}
    \date{}
    \predate{}\postdate{}
  

\usepackage{amsthm}
\newtheorem{theorem}{Theorem}[section]
\newtheorem{lemma}{Lemma}[section]
\newtheorem{corollary}{Corollary}[section]
\newtheorem{proposition}{Proposition}[section]
\newtheorem{conjecture}{Conjecture}[section]
\theoremstyle{definition}
\newtheorem{definition}{Definition}[section]
\theoremstyle{definition}
\newtheorem{example}{Example}[section]
\theoremstyle{definition}
\newtheorem{exercise}{Exercise}[section]
\theoremstyle{remark}
\newtheorem*{remark}{Remark}
\newtheorem*{solution}{Solution}
\let\BeginKnitrBlock\begin \let\EndKnitrBlock\end
\begin{document}

{
\setcounter{tocdepth}{2}
\tableofcontents
}
\hypertarget{scope-of-the-swim-package}{%
\section{Scope of the SWIM package}\label{scope-of-the-swim-package}}

\hypertarget{Rfunctions}{%
\subsection{Stressing a model}\label{Rfunctions}}

We define a \textbf{model} to consist of a random vector \(\boldsymbol{X} = (X_1, \ldots, X_n)\), the \textbf{model components} or \textbf{input factors}, a model function \(g \colon \mathbb{R}^n \to \mathbb{R}\), which gives, when applied to the input factors \(\boldsymbol{X}\), the one-dimensional random \textbf{output} of interest \(Y = g(\boldsymbol{X})\). Starting from a \textbf{baseline probability}, \(P\), a probability measure describing the current beliefs regarding (or software implementation of) the model, we are interested in \textbf{stressed models} that fulfil probabilistic constraints and have minimal Kullback-Leibler divergence. Specifically, a \textbf{stressed probability} \(Q\) is the solution to
\begin{equation} 
\min_{ Q \in \mathcal{P}} D_\text{KL}(Q \| P), \quad
\text{s.t. probabilistic constraints of the model under }Q \text{ hold},
\label{eq:optimisation}
\end{equation}
where the Kullback-Leibler divergence is given by \(D_\text{KL}(Q \| P) = \int \frac{dQ}{dP} \log \big(\frac{dQ}{dP} \big)\mathrm{d}P\), for \(Q\) absolutely continuous with respect to \(P\). The probabilistic constraints in \eqref{eq:optimisation} are with respect to (one of) the model inputs or the output and the implemented types of stresses are summarised in the subsequent table.

\begin{longtable}[]{@{}llll@{}}
\toprule
\begin{minipage}[b]{0.22\columnwidth}\raggedright
R function\strut
\end{minipage} & \begin{minipage}[b]{0.38\columnwidth}\raggedright
Stress\strut
\end{minipage} & \begin{minipage}[b]{0.09\columnwidth}\raggedright
\texttt{type}\strut
\end{minipage} & \begin{minipage}[b]{0.19\columnwidth}\raggedright
Reference\strut
\end{minipage}\tabularnewline
\midrule
\endhead
\begin{minipage}[t]{0.22\columnwidth}\raggedright
\texttt{stress()}\strut
\end{minipage} & \begin{minipage}[t]{0.38\columnwidth}\raggedright
wrapper for the \texttt{stress\_} functions\strut
\end{minipage} & \begin{minipage}[t]{0.09\columnwidth}\raggedright
\strut
\end{minipage} & \begin{minipage}[t]{0.19\columnwidth}\raggedright
\strut
\end{minipage}\tabularnewline
\begin{minipage}[t]{0.22\columnwidth}\raggedright
\texttt{stress\_user()}\strut
\end{minipage} & \begin{minipage}[t]{0.38\columnwidth}\raggedright
user defined scenario weights\strut
\end{minipage} & \begin{minipage}[t]{0.09\columnwidth}\raggedright
\texttt{user}\strut
\end{minipage} & \begin{minipage}[t]{0.19\columnwidth}\raggedright
\strut
\end{minipage}\tabularnewline
\begin{minipage}[t]{0.22\columnwidth}\raggedright
\texttt{stress\_prob()}\strut
\end{minipage} & \begin{minipage}[t]{0.38\columnwidth}\raggedright
probabilities of intervals\strut
\end{minipage} & \begin{minipage}[t]{0.09\columnwidth}\raggedright
\texttt{prob}\strut
\end{minipage} & \begin{minipage}[t]{0.19\columnwidth}\raggedright
Proposition \ref{prp:prop-prob-interval}\strut
\end{minipage}\tabularnewline
\begin{minipage}[t]{0.22\columnwidth}\raggedright
\texttt{stress\_mean()}\strut
\end{minipage} & \begin{minipage}[t]{0.38\columnwidth}\raggedright
means\strut
\end{minipage} & \begin{minipage}[t]{0.09\columnwidth}\raggedright
\texttt{mean}\strut
\end{minipage} & \begin{minipage}[t]{0.19\columnwidth}\raggedright
Proposition \ref{prp:prop-prob-moment}\strut
\end{minipage}\tabularnewline
\begin{minipage}[t]{0.22\columnwidth}\raggedright
\texttt{stress\_mean\_sd()}\strut
\end{minipage} & \begin{minipage}[t]{0.38\columnwidth}\raggedright
means and standard deviations\strut
\end{minipage} & \begin{minipage}[t]{0.09\columnwidth}\raggedright
\texttt{mean\ sd}\strut
\end{minipage} & \begin{minipage}[t]{0.19\columnwidth}\raggedright
Proposition \ref{prp:prop-prob-moment}\strut
\end{minipage}\tabularnewline
\begin{minipage}[t]{0.22\columnwidth}\raggedright
\texttt{stress\_moment()}\strut
\end{minipage} & \begin{minipage}[t]{0.38\columnwidth}\raggedright
moments, functions of moments\strut
\end{minipage} & \begin{minipage}[t]{0.09\columnwidth}\raggedright
\texttt{moment}\strut
\end{minipage} & \begin{minipage}[t]{0.19\columnwidth}\raggedright
Proposition \ref{prp:prop-prob-moment}\strut
\end{minipage}\tabularnewline
\begin{minipage}[t]{0.22\columnwidth}\raggedright
\texttt{stress\_VaR()}\strut
\end{minipage} & \begin{minipage}[t]{0.38\columnwidth}\raggedright
VaR risk measure, a quantile\strut
\end{minipage} & \begin{minipage}[t]{0.09\columnwidth}\raggedright
\texttt{VaR}\strut
\end{minipage} & \begin{minipage}[t]{0.19\columnwidth}\raggedright
Proposition \ref{prp:prop-risk-measure}\strut
\end{minipage}\tabularnewline
\begin{minipage}[t]{0.22\columnwidth}\raggedright
\texttt{stress\_VaR\_ES()}\strut
\end{minipage} & \begin{minipage}[t]{0.38\columnwidth}\raggedright
VaR and ES risk measures\strut
\end{minipage} & \begin{minipage}[t]{0.09\columnwidth}\raggedright
\texttt{VaR\ ES}\strut
\end{minipage} & \begin{minipage}[t]{0.19\columnwidth}\raggedright
Proposition \ref{prp:prop-risk-measure}\strut
\end{minipage}\tabularnewline
\bottomrule
\end{longtable}

\hypertarget{the-stress-function-and-the-swim-object}{%
\subsubsection{\texorpdfstring{The \texttt{stress} function and the \texttt{SWIM} object}{The stress function and the SWIM object}}\label{the-stress-function-and-the-swim-object}}

The \texttt{stress()} function is a wapper for the \texttt{stress\_} functions, with \texttt{stress(type\ =\ "type",\ ...\ )} and \texttt{stress\_type(...)} being equivalent. The \texttt{stress()} function solves optimisation \eqref{eq:optimisation} for constraints specified through \texttt{type}. For a data frame of realisations of the model inputs and corresponding output, provided via the parameter \texttt{x}, the \texttt{stress()} function calculates the scenario weights, such that under the new scenario weights, the constraints are fulfilled. The \texttt{stress()} functions returns a \texttt{SWIM} object containing a list of:

\begin{longtable}[]{@{}ll@{}}
\toprule
\endhead
\texttt{x} & the realistaions of the model input /output\tabularnewline
\texttt{new\_weights} & the scenario weights\tabularnewline
\texttt{type} & the ``type'' of the stress\tabularnewline
\texttt{specs} & the details about the stress\tabularnewline
\bottomrule
\end{longtable}

The data frame containing the underlying realisations of the model input and corresponding output of a \texttt{SWIM} object (\texttt{x} in the above table), can be extracted using \texttt{get\_data()}. Similarly, \texttt{get\_weights()} and \texttt{get\_weightsfun()} provide the scenario weights, respectively the functions, that when applied to the data generate the scenarion weights. The specification of the applied stress can be obtained using \texttt{get\_specs()}.

\hypertarget{stressing-disjoint-probability-intervals}{%
\subsubsection{Stressing disjoint probability intervals}\label{stressing-disjoint-probability-intervals}}

Optimisation \eqref{eq:optimisation} with constraints consisting of disjoint probability intervals, @Csiszar1975, @Cambou2017, allow to define stresses by altering regions or events of a model component. The scenario weights for a stress according to Proposition \ref{prp:prop-prob-interval} is implemented via the function \texttt{stress\_prob()}, or equivalently \texttt{stress(type\ =\ "prob",\ ...)}, and the stressed probability intervals are specified through the \texttt{lower} and \texttt{upper} endpoints of the intervals.

\BeginKnitrBlock{proposition}[see Proposition 3.1 in @Pesenti2019]
\protect\hypertarget{prp:prop-prob-interval}{}{\label{prp:prop-prob-interval} \iffalse (see Proposition 3.1 in @Pesenti2019) \fi{} }
Let \(B_1, \ldots, B_I\) be disjoint sets with \(P(Y \in B_i) >0\), for all \(i = 1, \ldots, I\), and \(\alpha_1, \ldots, \alpha_I > 0\)\footnote{For simpilcity of presentation, we henceforth stress the output of a model, however, a stress can be applied to the model output as well as model components.} such that \(\alpha_1 + \ldots + \alpha_I < 1\). Then the solution to
\begin{equation} 
\min_{Q} D_\text{KL}(Q \| P), \quad
\text{s.t. } Q(Y \in B_i) = \alpha_i, ~i = 1, \ldots, I.
\end{equation}
is discribed by the change-of-measure (Radon-Nikodym-density) from \(P\) to \(Q\)
\begin{equation} 
w  = \sum_{i = 0}^I \frac{\alpha_i}{P(Y \in B_i)}1_{ \{Y \in B_i
\}},
\end{equation}
where \(\alpha_0 = 1 - \alpha_1 + \ldots + \alpha_I\) and \(B_0 = \Big( \bigcup_{i = 1}^I B_i \Big)^c\).
\EndKnitrBlock{proposition}

Note that the scenario weights, \(w\), are a function of the stressed model component, \(Y\).

\hypertarget{stressing-moments}{%
\subsubsection{Stressing moments}\label{stressing-moments}}

The functions \texttt{stress\_mean()}, \texttt{stress\_mean\_sd()} and \texttt{stress\_moment()} can be applied to multiple model components and are the only \texttt{stress} functions that have scenrio weights not given in analytical form. Thus, there is no guarantee of existence of the stressed model. Specifically, \texttt{stress\_mean()} solves optimisation in Proposition \ref{prp:prop-prob-moment} via numerical optimisation using the \href{https://CRAN.R-project.org/package=nleqslv}{nleqslv} package.

\BeginKnitrBlock{proposition}
\protect\hypertarget{prp:prop-prob-moment}{}{\label{prp:prop-prob-moment} }For \(i = 1, \ldots, I\), let \(J_i \subset \{1, \ldots, n\}\) and \(f_i \colon \mathbb{R}^{J_i} \to \mathbb{R}\) be functions and consider
\begin{equation} 
\min_{Q} D_\text{KL}(Q \| P), \quad
\text{s.t. } E^Q(f_i(X_{J_i}) ) = m_i, ~i = 1, \ldots, I.
\end{equation}
where \(E^Q(\cdot)\) denotes the expectation under the stressed model.
\EndKnitrBlock{proposition}

\hypertarget{stressing-risk-measures}{%
\subsubsection{Stressing risk measures}\label{stressing-risk-measures}}

Recall that the \(\text{VaR}\) at level \(0 < \alpha < 1\) of a random variable \(Z\) with distrbution \(F\), is defined as the \(\alpha-\)quantile of \(F\), that is \[\text{VaR}_\alpha(Z) = F^{-1}(\alpha).\] The \(\text{ES}\) at level \(0 < \alpha < 1\) of a random variable \(Z\) is given by \[\text{ES}_\alpha(Z) = \int_0^1 \text{VaR}_u(Z) \mathrm{d}u.\]

\BeginKnitrBlock{proposition}[see Propositions 3.2 and 3.3 in @Pesenti2019]
\protect\hypertarget{prp:prop-risk-measure}{}{\label{prp:prop-risk-measure} \iffalse (see Propositions 3.2 and 3.3 in @Pesenti2019) \fi{} }Let \(0< \alpha <1\) and \(q, s \in \mathbb{R}\) such that \(\text{VaR}_{\alpha}(Y)<q < s\), and consider the optimisations
\begin{align}
i) \quad & \min_{ Q \ll \mathcal{P}} D_\text{KL}(Q \| P), \quad
\text{s.t. } \text{VaR}_{\alpha }^Q(Y) = q;\\
ii) \quad & \min_{ Q \ll \mathcal{P}} D_\text{KL}(Q \| P), \quad
\text{s.t. } \text{VaR}_{\alpha }^Q(Y) = q, \text{ ES}_{\alpha }^Q(Y) = s,
\end{align}
where \(\text{VaR}^Q\) and \(\text{ES}^Q\) denote the \(\text{VaR}\) and the \(\text{ES}\) under the stressed model.
The change-of-measure (Radon-Nikodym-density) of the solution is then described by
\begin{align}
i) \quad & 
w  =    \frac{\alpha }{  P( Y < q)} 1_{\{ Y < q\}}  + \frac{1- \alpha }{ P(Y \geq q)} 1_{\{  Y \geq q\}};\\
ii) \quad & 
w  =    \frac{\alpha }{  P( Y < q)} 1_{\{ Y < q\}}  + \frac{1- \alpha }{E\left( e^{ \theta (Y - q) } Y \geq q \right)} 1_{\{  Y \geq q\}} e^{\theta (Y - q)}, \text{ for a suitable}; \, \theta >0.\\
\end{align}
\EndKnitrBlock{proposition}

\hypertarget{visual-tools}{%
\subsection{Visual tools}\label{visual-tools}}


\end{document}
